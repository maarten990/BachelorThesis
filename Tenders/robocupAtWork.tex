\documentclass[a4paper, 10pt]{article}
\usepackage[english]{babel}

\author{Maarten de Jonge}
\title{Recognizing objects for the RoboCup@Work League \\
       \Large{Supervisor: Arnoud Visser}}

\begin{document}
\maketitle

\section*{Motivation}
I am interested in robotics, specifically the aspects of kinematics and computer
vision. My affinity with the RoboCup competition makes this project a natural
choice.

\section*{Skills}
I have prior experience in robotics (specifically, the Nao robot) through
participation in the Dutch Nao Team. I am comfortable with high-performance
programming (specifically C and C++) and implementation of mathematical
algorithms. In addition, I am reasonably skilled in the forms of mathematics
that generally accompany robotics (most importantly linear algebra).

\section*{Problem Approach}
The inverse kinematics of the type of arm on the Kuka YouBot is a generally
solved problem, which boils this problem down to one of object recognition. Any
kind of motion should be trivial after becoming familiar with the robot's
programming interface. 

How to handle the object recognition depends on the types of objects and the
type of camera used. In any case, the way to start is by doing literature study
and seeing if previously used algorithms can be adapted to this specific task.

\section*{Expected Result}
This project consists of a number of a rather popular sub-problems (mainly
object recognition). Because of this, I expect there to be a good amount of
literature available on the subject which gives the project a very high
probability of coming to succesful completion.
	
\end{document}
