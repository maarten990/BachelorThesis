\documentclass[a4paper, 12pt]{article}
\usepackage[english]{babel}

\author{Maarten de Jonge}
\title{Projective Geometric Algebra}

\begin{document}
\maketitle

\section*{Motivation}
I am generally drawn to the more technical/mathematical side of artificial
intelligence, which makes this project (leaning more to the mathematical side)
interesting to me. The challenging nature and cutting-edgeness of the research
is also attractive.

\section*{Skills}
I am familiar with the basics of geometric algebra through the honours course. I
am generally fast to pick up new abstract concepts, and given the time available
for the project I am confident in my ability to quickly master the required
mathematics.

Furthermore, I am comfortable with low level, high performance computing
(notably C and C++), as can come in handy for a potential implementation of
the project.

\section*{Problem Approach}
It is hard to be concrete in this regard due to the open-ended nature of the
problem statement. The most sensible way to approach the problem would be to
first familiarize myself with Patrick's previous work and the required
mathematical background, then evaluate the possible ways to proceed.

\section*{Expected Result}
This is again hard to predict; possibilities include new insights into the
properties of line space or software implementations of known properties.
	
\end{document}
