\documentclass[a4paper, 10pt]{article}
\usepackage[english]{babel}
\usepackage{hyperref}

\author{Maarten de Jonge}
\title{Assignment 3 - Literature}

\begin{document}
\maketitle

\section{}
\subsection{Learning and Reasoning with Action-Related Places for Robust Mobile Manipulation}
\subsubsection*{Research Question}
Where should a robot position itself to have the highest chance of success in a
grasping task?
\subsubsection*{Conclusion}
The ARPLace system works well and learns compact models.
\subsubsection*{Type of research question}
The work done is rather formal, but the evaluation is done empirically.
\subsubsection*{Type of research}
Implementing and testing a method.

\subsection{Robust Local Search for Solving RCPSP/max with Durational Uncertainty}
\subsubsection*{Research Question}
Scheduling for tasks with durational uncertainty of activities, along with
temporal dependencies.
\subsubsection*{Conclusion}
The proposed method performs better than existing methods.
However, it is unable to cope with global factors, e.g. weather conditions.
\subsubsection*{Type of research question}
The work done is rather formal, but the evaluation is done empirically.
\subsubsection*{Type of research}
Implementing and testing a method.

\subsection{Location-Based Reasoning about Complex Multi-Agent Behavior}
\subsubsection*{Research Question}
There are 4 explicitly named research questions related to modeling multi-agent behaviour.
I might as well simply quote them:

\begin{enumerate}
\item Can we reliably recognize complex multi-agent activities in the CTF
  dataset even in the pres- ence of severe noise?
\item Can models of attempted activities be automatically learned by leveraging
  existing models of successfully performed actions?
\item Does modeling both success and failure allow us to infer the respective
  goals of the activities?
\item Does modeling failed attempts of activities improve the performance on
  recognizing the ac- ivities themselves?
\end{enumerate}

\subsubsection*{Conclusion}
Their method, using Markov logic, is more accurate on real-world data than other
methods. It is however more computationally expensive. Modeling failed attempts
of activities does improve performance.

\subsubsection*{Type of research question}
Implementing and testing a method.
\subsubsection*{Type of research}
Empirical testing with a formal background.

\subsection{The CQC Algorithm: Cycling in Graphs to Semantically Enrich and Enhance a Bilingual Dictionary}
\subsubsection*{Research Question}
Automated disambiguation of translations in bilingual dictionaries.

\subsubsection*{Conclusion}

\begin{enumerate}
  \item The approach using CQC enables state of the art performance
    in dictionary disambiguation.
  \item 
  \item CQC obtains the best result in synonym extraction among lexicon-based
    systems, although performs worse than data-intensive approaches.
\end{enumerate}

\subsubsection*{Type of research question}
Implementing and testing a method.

\subsubsection*{Type of research}
Formal algorithm with empirical testing.

\subsection{Counting-Based Search: Branching Heuristics for Constraint Satisfaction Problems}
\subsubsection*{Research question}
Designing a counting-based heuristic for constraint satisfaction problems.
\subsubsection*{Conclusion}
Counting-based search generally outperforms other heuristics.

\subsubsection*{Type of research question}
Implementing and testing a method.

\subsubsection*{Type of research}
Formal algorithm with empirical testing.
\end{document}
