\documentclass[a4paper, 10pt]{article}
\usepackage[english]{babel}
\usepackage[utf8]{inputenc}
\usepackage{amsfonts}
\usepackage{graphicx}
\usepackage[hidelinks]{hyperref}

\author{Maarten de Jonge}
\title{2D Projective Geometric Algebra - Literature Review}

\begin{document}
\newcommand{\rp}{$\mathbb{R}^{3,3}$ }

\maketitle

\section{Literature Review}
Li and Zhang\cite{hangbo2011} model line geometry in a 6D space, representing
lines by means of their Pl\"{u}cker coordinates (with the constraint that the
vector must square to zero, i.e. be a null vector).  The 6D vector of a line's
Pl\"{u}cker coordinates corresponds to the coordinates of of the line's
direction and moment on a basis of 2-blades in $\wedge^2(V^4)$, a bivector space
over the 4D homogeneous vector space. Through a clever correspondence with the
homogeneous space, Li and Zhang also provide a metric for the 6D
representational space and show that it has the signature $\mathbb{R}^{3, 3}$. They
proceed to give geometric interpretations of the various blades that can be
formed in $\mathbb{R}^{3, 3}$.

Pottmann and Wallner\cite{pottmann2001computational} do similarly, except they
stay within the domain of linear algebra, giving a different viewpoint to what is
essentially the same geometry.

Leo Dorst\cite{dorst2013versors} provides visual representations for many of the
geometric features described by \cite{hangbo2011} and
\cite{pottmann2001computational} and describes how to extract the data required
for implementing these visualizations in software. Additionally, the paper
contains a novel approach to modeling 2D projective geometry in the previously
described \rp, defining the elementary operations (translations, scaling,
perspective transformations, a rotation and a squeeze) in terms of their
transformation matrices.

GAViewer\footnote{\url{http://www.geometricalgebra.net/gaviewer\_download.html}}
is a graphical calculator for geometric algebra which handles many different
algebraic models. Patrick de Kok\cite{dekok2012} implemented the
Pl\"{u}cker model of Li and Zhang in GAViewer and added many of the
visualizations described by Dorst.

Cinderella\cite{richter1999interactive} is a software package for performing
interactive 2D projective geometry using a 6D homogeneous representation with
complex numbers. This avoids the edge cases commonly experienced in classic
Euclidean geometry. For a good read on their approach to this problem, see
\url{http://doc.cinderella.de/tiki-index.php?page=Theoretical+Background}.
Note that their use of 6 dimensions to represent 2D geometry hints at a
correspondence with the model described by Dorst, who also uses 6 (albeit
different) dimensions for representing 2D projective geometry.

\bibliographystyle{plain}
\bibliography{library}
\end{document}
