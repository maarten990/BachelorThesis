\documentclass[a4paper, 12pt]{article}
\usepackage[english]{babel}
\usepackage{hyperref}

\author{Inge Becht \\ Maarten de Jonge}
\title{Assignment 2 - What is AI research?}

\begin{document}
\maketitle

\section{Empirical}
Example: ``Fault Detection Using Support Vector Machines and Artificial Neural
Networks, Augmented By Genetic Algorithms'',
\url{ttp://www.sciencedirect.com/science/article/pii/S0888327001914542#}
In this paper a comparison is made of two different machine learning algorithms,
Support Vector Machines and Neural Networks, to see which one works the best in
detecting faults in Rotation Machinery. Also both systems were extended on with genetic algorithms to
see its potential. The way the algorithms were in the end tested was by using a
train and test set on the specific domain. The conclusion is that Artificial
Neural Networks are faster to train on the domain and gives an overall better
performance. 

We can call this kind of AI research emperical as it uses the results specific
on the domain to conclude which method is better for it. It would have been
formal if instead of these experiments, reasoning would have been used about the
machine learning approaches to determine the best possible outcome.


\section{Formal}

\section{Design}
Example: ``Humanoid robot localisation using stereo vision'',
\url{http://ieeexplore.ieee.org/xpls/abs\_all.jsp?arnumber=1573539}.
This research is about implementing humanoid robot localisation by means of
stereo vision. Stereo vision has previously been used for localisation of
wheeled robots, but poses additional challenges for humanoids due to the jerky
motions generated by humanoid locomotion.
This is design research because it makes use of previously developed methods for
localisation, but applies them to a new domain (humanoid robots instead of
wheeled ones).

The conclusion is that the approach works, but is very noisy and will require
more computationally intensive or accurate methods for practical tasks such as
footstep planning.
	
\end{document}
