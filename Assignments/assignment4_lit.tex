\documentclass[a4paper, 10pt]{article}
\usepackage[english]{babel}
\usepackage[utf8]{inputenc}
\usepackage{amsfonts}
\usepackage[hidelinks]{hyperref}

\author{Maarten de Jonge}
\title{Assignment 4 - Literature and Research Question \\
\large{Projective Geometric Algebra}}

\begin{document}
\maketitle

\section*{Research Question}
In the previous year, Patrick de Kok did a project about projective line
geometry for geometric algebra\cite{dekok2012} in which he extended the GAViewer
software\footnote{\url{http://www.geometricalgebra.net/gaviewer\_download.html}}
with visualizations for the Pl\"{u}cker model of geometric algebra. Some
visualizations, such as those of 3-blades, were not yet implemented due to a
lack of understanding at the time. Recent developments allow for these
visualizations to now be implemented, which I will be doing.

The further direction of the project is still up for consideration. One of the
options is to apply projective geometric algebra to 2D projective geometry in
the same vein as the Cinderella
software\footnote{\url{http://doc.cinderella.de/tiki-index.php?page=Theoretical+Background}},
and compare the results (in terms of e.g. performance, ease of implementation,
elegance, or other such metrics) of the geometric algebra approach with the
complex linear algebra used by Cinderella.

The next weeks will bring more clarity as I get more acquainted with the subject
and a more concrete direction can be formulated.

\section*{Literature}

\begin{itemize}
  \item Patrick de Kok's bachelor thesis from last year\cite{dekok2012}, in which
    he provides an introduction to projective line geometry/Pl\"{u}cker
    coordinates and describes the implementation of their visualizations in
    GAViewer, provides a good starting point for getting acquainted with
    previous work before extending it.
  \item \emph{Geometric Algebra for Computer
    Scientists}\cite{dorst2009geometric} is an introductory book to geometric
    algebra which describes various models of geometric algebra, including the
    homogeneous and Pl\"{u}cker models which provide most of the necessary
    background to understand the rest of the literature.
  \item Li and Zhang\cite{hangbo2011} model classical line geometry in
    $\mathbb{R}^{3,3}$ using Pl\"{u}cker coordinates, which lays the basis
    for projective geometry using geometric algebra. This paper is an important
    basis for the work done by Patrick, and thus will aid in understanding his
    previous work.
  \item Pottmann and Wallner\cite{pottmann2001computational} in chapters 2 and 3
    of their book describe the geometric elements of 6D space with Pl\"{u}cker
    coordinates in linear algebra.. Although using linear algebra, there is a
    correspondence with the geometric algebra described by \cite{hangbo2011}
    which makes this helpful in describing these elements in geometric
    algebra.
  \item An unreleased paper by Leo Dorst\cite{dorst2013versors} expands on the
    things done by Patrick last year with new insights, such as extraction of
    parameters of the \emph{regulus} which allow it to be visualised in
    GAViewer. It also details 2D projective geometry, one of the potential
    directions to take in this project.
\end{itemize}

\bibliographystyle{plain}
\bibliography{library}
\end{document}
